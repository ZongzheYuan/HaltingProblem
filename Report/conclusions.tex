\chapter{Evaluation, Summary and Futher Work}

\section{Evaluation to the Project}

\section{Summary and Conclusion}

\section{Future Work}
\subsection{WHILE Program Variables Transformation}
The universal \WHILE program defined in \ref{universal-I} could only simulate the \WHILE program that has only one variable.
Thus the final proof could only say that a machine \wit{h} which has only one variable that could \wit{decide} the \wit{halting problem} is \wit{false}.
However the \WHILE program that has only one variable (we name it as \textbf{WHILE-I} program) has the same computation ability compare to the \WHILE program that has many variables.
That means, the number of variable to the \WHILE program doesn't matter, and doesn't violate the property of \wit{Turing-completeness} to the \WHILE program.
However, because \Agda has strict type, the program must know its variable number before it has been defined.
Thus, we can construct some rules to transform the \WHILE program to \textbf{WHILE-I} program and prove that they have the same effect, 
which means for \wit{p} $\in$ \md{P} $n$ and \wit{input}, \wit{output} $\in$ \md{D}, $p(input)$ yielding $output$ implies that $\exists$ $p-I$ $\in$ \md{P} $1$, $p-I(input)$ yielding $output$.\\
Initially we should transform the \wit{environment} of the program from \textbf{Vec D n} to \textbf{Vec D 1}, which means accumulate all the variables in the first \wit{environment} to the first element in the second \wit{environment} by the operator $\cdot$.
For example, initially we have \wit{environment} of five variables: (\wit{A :: B :: C :: D :: E ::[]}), then the transformation will construct a tree structure data for the first element of the new \wit{environment}:\\\\
\begin{tikzpicture}[sibling distance=10em,
  every node/.style = {shape=rectangle,
    draw, align=center,
    top color=white, bottom color=white}]]
  \node {$\cdot$}
    child { node {\wit{A}} }
    child { node {$\cdot$}
      child { node {\wit{B}}}
      child { node {$\cdot$} 
	child { node {\wit{C}} } 
	child { node {$\cdot$} 
		child { node {\wit{D}} }
		child { node {$\cdot$} 
			child { node {\wit{E}} }
			child { node {\wit{dnil}} } } } } };
\end{tikzpicture} \textbf{\wit{:: []}}\\
Then we can transform the \wit{expression}. 
For the \wit{expression} of \wit{head}, \wit{tail}, \wit{cons}, \wit{nil} and \wit{equality}, we can easily recursively transform the target \wit{expression} from the argument.
For the \wit{expression} that use the value of variable that in the \wit{environment}, for example \wit{var C} in our previous example, we can use the \wit{expression} \wit{head (tail (tail (var zero)))} to get the same value from the transformed \wit{1} variable environment.\\
Then similarly we can transform the \wit{command}.
For the \wit{command} of \wit{sequence} and \wit{while} loop, we can easily recursively transform the target \wit{command} from the argument.
For the \wit{expression} of \wit{assignment}, for example we assign variable \wit{C} with value \wit{K} $\in$ \md{E}, the \wit{environment} with five variables updated as:
\begin{center}
(\wit{A :: B :: C :: D :: E ::[]}) $\Longrightarrow$ (\wit{A :: B :: \textbf{K} :: D :: E ::[]})
\end{center}
Then we can write the new \wit{assignment} \wit{command} as \\
$zero$ $:=$ $(hd$ $(var$ $zero))$ $\cdot$ $((hd$ $(tl$ $(var$ $zero)))$ $\cdot$ $((K)$ $\cdot$ $(tl$ $(tl$ $(tl$ $(var$ $zero))))))$ and\\\\
\begin{tikzpicture}[sibling distance=4em,
  every node/.style = {shape=rectangle,
    draw, align=center,
    top color=white, bottom color=white}]]
  \node {$\cdot$}
    child { node {\wit{A}} }
    child { node {$\cdot$}
      child { node {\wit{B}}}
      child { node {$\cdot$} 
	child { node {\wit{C}} } 
	child { node {$\cdot$} 
		child { node {\wit{D}} }
		child { node {$\cdot$} 
			child { node {\wit{E}} }
			child { node {\wit{dnil}} } } } } };
\end{tikzpicture} \textbf{\wit{:: []}} $\Longrightarrow$
\begin{tikzpicture}[sibling distance=4em,
  every node/.style = {shape=rectangle,
    draw, align=center,
    top color=white, bottom color=white}]]
  \node {$\cdot$}
    child { node {\wit{A}} }
    child { node {$\cdot$}
      child { node {\wit{B}}}
      child { node {$\cdot$} 
	child { node {\wit{\textbf{K}}} } 
	child { node {$\cdot$} 
		child { node {\wit{D}} }
		child { node {$\cdot$} 
			child { node {\wit{E}} }
			child { node {\wit{dnil}} } } } } };
\end{tikzpicture} \textbf{\wit{:: []}}\\
Finally we can transform the \wit{program}.
We should transform the initial \wit{environment} at first.
Then we can transform the \wit{command}.
Finally we should get the result from the transformed \wit{environment}.\\
The proof of the correctness of the transformation will be held in the future.
\subsection{Interpret WHILE Program with Arbitrary Variables}
If we can prove that the \WHILE program has the same computation ability with \textbf{WHILE-I} program which has only one variable,
we could conclude that our universal \WHILE program defined in \ref{universal-I} could simulate the \WHILE program has arbitrary variables.
For example, for \wit{p} $\in$ \md{P} $n$ and \wit{input}, \wit{output} $\in$ \md{D}, if we want to use our universal \WHILE program to simulate $p(input)$,
then we could transform $p$ to $p-I$ which has only one variable by our predefined transformation function.
And we know that $p(input)$ yielding $output$ implies that $p-I(input)$ yielding $output$ by the proof of correctness of the transformation function.
And by the proof of the correctness of the universal \WHILE program we know that $p-I(input)$ yielding $output$ implies that $u(\lfloor p-I\rfloor \cdot input)$ yielding $output$.
Thus we can conclude that for \wit{p} $\in$ \md{P} $n$ and \wit{input}, \wit{output} $\in$ \md{D}, $p(input)$ yielding $output$ implies $u(\lfloor p-I\rfloor \cdot input)$ yielding $output$ which means universal \WHILE program defined in \ref{universal-I} could simulate the \WHILE program has arbitrary variables.\\
Thus for the proof of \wit{halting problem}, we could say that for \wit{p} $\in$ \md{P} $n$, \wit{h} could decide the \wit{halting problem} implies \wit{false}.

\iffalse
if-else8 ((hd (hd (var Cd))) =? quoteE)
                   ((St := (cons (tl (hd (var Cd))) (var St)))
                   →→ (Cd := tl (var Cd)))
         (if-else8 (hd (hd (var Cd)) =? varE)
                   (if8 (tl (hd (var Cd)) =? valueE zero)
                        ((Cd := tl (var Cd))
                        →→ (St := cons (var V1) (var St))))
         (if-else8 (hd (hd (var Cd)) =? hdE)
                   (Cd := cons (tl (hd (var Cd)))
                               (cons dohdE (tl (var Cd))))
         (if-else8 (hd (var Cd) =? dohdE)
                   ((Cd := tl (var Cd))
                   →→ (St := cons (hd (hd (var St))) (tl (var St))))
         (if-else8 (hd (hd (var Cd)) =? tlE)
                   (Cd := cons (tl (hd (var Cd)))
                               (cons dotlE (tl (var Cd))))
         (if-else8 (hd (var Cd) =? dotlE)
                   ((Cd := tl (var Cd))
                   →→ (St := cons (tl (hd (var St))) (tl (var St))))
         (if-else8 (hd (hd (var Cd)) =? consE)
                   (Cd := cons (hd (tl (hd (var Cd))))
                               (cons (tl (tl (hd (var Cd))))
                                     (cons doconsE (tl (var Cd)))))
         (if-else8 (hd (var Cd) =? doconsE)
                   ((Cd := tl (var Cd))
                   →→ (St := cons (cons (hd (tl (var St)))
                                  (hd (var St))) (tl (tl (var St)))))
         (if-else8 (hd (hd (var Cd)) =? =?E)
                   (Cd := cons (hd (tl (hd (var Cd))))
                               (cons (tl (tl (hd (var Cd))))
                                     (cons do=?E (tl (var Cd)))))
         (if-else8 (hd (var Cd) =? do=?E)
                   ((Cd := tl (var Cd))
                   →→ (St := cons ((hd (tl (var St))) =? (hd (var St)))
                                  (tl (tl (var St)))))
         (if-else8 (hd (hd (var Cd)) =? →→E)
                   (Cd := cons (hd (tl (hd (var Cd))))
                               (cons (tl (tl (hd (var Cd)))) (tl (var Cd))))
         (if-else8 (hd (hd (var Cd)) =? :=E)
                   (if8 ((hd (hd (tl (hd (var Cd))))) =? varE)
                        (if8 (tl (hd (tl (hd (var Cd)))) =? valueE zero)
                             (Cd := cons (tl (tl (hd (var Cd))))
                                         (cons doasgnE (tl (var Cd))))))
         (if-else8 (hd (var Cd) =? doasgnE)
                   ((Cd := tl (var Cd))
                   →→ ( (V1 := hd (var St)) →→ (St := tl (var St))))
         (if-else8 (hd (hd (var Cd)) =? whileE)
                   (Cd := cons (hd (tl (hd (var Cd))))
                               (cons dowhE
                                     (cons (cons whileE
                                                 (cons (hd (tl (hd (var Cd))))
                                                       (tl (tl (hd (var Cd))))))
                                           (tl (var Cd)))))
         (if-else8 (hd (var Cd) =? dowhE)
                   (if8 (hd (hd (tl (var Cd))) =? whileE)
                        (if-else8 (hd (var St) =? nil)
                                  ((Cd := tl (tl (var Cd)))
                                  →→ (St := tl (var St)))
                        ((Cd := cons (tl (tl (hd (tl (var Cd)))))
                                     (cons (cons whileE
                                                 (cons (hd (tl (hd (tl (var Cd)))))
                                                       (tl (tl (hd (tl (var Cd)))))))
                                           (tl (tl (var Cd)))))
                        →→ (St := tl (var St)))))
         (if8      (var Cd =? nil)                (Cd := nil))))))))))))))))
\fi