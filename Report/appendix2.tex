\chapter{Definition of the Universal WHILE model}
The definition of the universal \WHILE language used in the project follows the definition in the paper \wit{Computability and complexity: from a programming perspective} by \wit{Neil D. Jones} in 1997\cite{jones_computability_1997}.
\section{Constant in WHILE}\label{appendix:const of d}
First of all, we define a function that would construct some value in data of \md{D} based on the natural number \md{N}:
\begin{code}
const : (n : ℕ) → D
const zero = dnil
const (suc n) = (dnil ∙ dnil) ∙ const n
\end{code}
Then, there are sixteen constants in \md{D} that are used to indicate special meaning in the universal \WHILE program.
\begin{code}[fontsize=\footnotesize]
dquote : D
dquote = const 1

d:= : D
d:= = const 2

d→→ : D
d→→ = const 3

dwhile : D
dwhile = const 4

dvar : D
dvar =  const 5

ddnil : D
ddnil = const 6

dcons : D
dcons = const 7

dhd : D
dhd = const 8

dtl : D
dtl = const 9

d=? : D
d=? = const 10

dohd : D
dohd = const 11

dotl : D
dotl = const 12

docons : D
docons = const 13

doasgn : D
doasgn = const 14

dowh : D
dowh = const 15

do=? : D
do=? = const 16
\end{code}
\section{Code the WHILE program}\label{appendix:code while}
The mapping function from the \WHILE program to \WHILE data is defined as:\\
\indent\hspace{3cm}\underline{read V$_i$; C; write V$_j$}\hspace{0.5cm} = \hspace{0.5cm} ((vari)\underline{C}(varj))\\\\
\indent\hspace{3cm}\underline{C; D}\hspace{3.3cm} = \hspace{0.5cm} (;\underline{CD})\\
\indent\hspace{3cm}\underline{while E do C}\hspace{1.8cm} = \hspace{0.5cm} (while\underline{EC})\\
\indent\hspace{3cm}\underline{V$_i$:= E}\hspace{2.9cm} = \hspace{0.5cm} (:=(vari)\underline{E})\\\\
\indent\hspace{3cm}\underline{V$_i$}\hspace{3.75cm} = \hspace{0.5cm} (vari)\\
\indent\hspace{3cm}\underline{d}\hspace{3.95cm} = \hspace{0.5cm} (quoted)\\
\indent\hspace{3cm}\underline{cons E F}\hspace{2.6cm} = \hspace{0.5cm} (cons\underline{EF})\\
\indent\hspace{3cm}\underline{hd E}\hspace{3.33cm} = \hspace{0.5cm} (hd\underline{E})\\
\indent\hspace{3cm}\underline{tl E}\hspace{3.5cm} = \hspace{0.5cm} (tl\underline{E})\\
\indent\hspace{3cm}\underline{=? E F}\hspace{2.8cm} = \hspace{0.5cm} (=?\underline{EF})\\
\section{Variable of the Universal WHILE Program}\label{appendix:variable of u}
We can define the eight variables of the universal \WHILE program as:
\begin{code}[fontsize=\footnotesize]
PD : Fin 8
PD = zero

Pp : Fin 8
Pp = suc zero

Cc : Fin 8
Cc = suc (suc (zero))

Cd : Fin 8
Cd = suc (suc (suc (zero)))

St : Fin 8
St = suc (suc (suc (suc (zero))))

V1 : Fin 8
V1 = suc (suc (suc (suc (suc (zero)))))

W : Fin 8
W = suc (suc (suc (suc (suc (suc (zero))))))

Z : Fin 8
Z = suc (suc (suc (suc (suc (suc (suc (zero)))))))
\end{code}
\section{Universal WHILE Program}\label{appendix:universal while}
The universal \WHILE program which could simulate the \WHILE program of $1$ variable is defined as:
\begin{code}
read PD;             (* Input (p.d) *)
  P := hd PD;        (* P = ((var 1) C (var 1)) *)
  C := hd (tl P)     (* C = hd tl p program code is C *)
  Cd := cons C nil;  (* Cd = (c.nil), Code to execute is c *)
  St := nil;         (* St = nil, Stack empty *)
  Vl := tl PD;       (* Vl = d Initial value of var.*)
  while Cd do STEP;  (* do while there is code to execute *)
write Vl;
\end{code}
Where the \wit{STEP} is the Macro that simulates the program.
\section{STEP Macro}\label{appendix:step}
When we are simulating a \WHILE program which has only one variable, we can easily define a stack machine as a Macro based on three variables: \wit{Cd} which is the \wit{command}, \wit{St} which is the stack and \wit{V1} which is the only one variable.
Here we define the syntax sugar \wit{cons$^{*}$} as \wit{cons$^{*}$} \wit{A B C} = \wit{cons} \wit{A} (\wit{cons B C}). 
Then we could rewrite [Cd, St] by:\\
\indent\hspace{0.5cm}[((quote D)$\cdot$Cr), St]\hspace{1.55cm} $\Rightarrow$ \hspace{0.5cm} [Cr, cons D St]\\
\indent\hspace{0.5cm}[((var 1)$\cdot$Cr), St]\hspace{2.1cm} $\Rightarrow$ \hspace{0.5cm} [Cr, cons V1 St]\\
\indent\hspace{0.5cm}[((hd E)$\cdot$Cr), St]\hspace{2.15cm} $\Rightarrow$ \hspace{0.5cm} [cons$^{*}$ E dohd Cr, Sr]\\
\indent\hspace{0.5cm}[(dohd$\cdot$Cr), (T$\cdot$Sr)]\hspace{1.7cm} $\Rightarrow$ \hspace{0.5cm} [Cr, cons (hd T) St]\\
\indent\hspace{0.5cm}[((tl E)$\cdot$Cr), St]\hspace{2.35cm} $\Rightarrow$ \hspace{0.5cm} [cons$^{*}$ E dotl Cr, St]\\
\indent\hspace{0.5cm}[(dotl$\cdot$Cr), (T$\cdot$Sr)]\hspace{1.9cm} $\Rightarrow$ \hspace{0.5cm} [Cr, cons (tl T) Sr]\\
\indent\hspace{0.5cm}[((cons E$_1$ E$_2$)$\cdot$Cr), St]\hspace{1.1cm} $\Rightarrow$ \hspace{0.5cm} [cons$^{*}$ E$_1$ E$_2$ docons Cr, St]\\
\indent\hspace{0.5cm}[(docons$\cdot$Cr), (U$\cdot$(T$\cdot$Sr))]\hspace{0.7cm} $\Rightarrow$ \hspace{0.5cm} [Cr, cons (cons T U) Sr]\\
\indent\hspace{0.5cm}[((=? E$_1$ E$_2$)$\cdot$Cr), St]\hspace{1.35cm} $\Rightarrow$ \hspace{0.5cm} [cons$^{*}$ E$_1$ E$_2$ do=? Cr, St]\\
\indent\hspace{0.5cm}[(do=?$\cdot$Cr), (U$\cdot$(T$\cdot$Sr))]\hspace{1cm} $\Rightarrow$ \hspace{0.5cm} [Cr, cons (=? T U) St]\\
\indent\hspace{0.5cm}[((; C$_1$ C$_2$)$\cdot$Cr), St]\hspace{1.8cm} $\Rightarrow$ \hspace{0.5cm} [cons$^{*}$ C$_1$ C$_2$ Cr, St]\\
\indent\hspace{0.5cm}[((:= (var 1) E)$\cdot$Cr), St]\hspace{0.9cm} $\Rightarrow$ \hspace{0.5cm} [cons$^{*}$ E doasgn Cr, St]\\
\indent\hspace{0.5cm}[(doasgn$\cdot$Cr), (W$\cdot$Sr)]\hspace{1.35cm} $\Rightarrow$ \hspace{0.5cm} \{Cd := Cr, St := Sr; V1 := W\}\\
\indent\hspace{0.5cm}[((while E C)$\cdot$Cr), St]\hspace{1.35cm} $\Rightarrow$ \hspace{0.5cm} [cons$^{*}$ E dowh (while E C) Cr, St]\\
\indent\hspace{0.5cm}[(dowh$\cdot$((while E C)$\cdot$Cr)), (nil$\cdot$Sr)]\hspace{0.85cm} $\Rightarrow$ \hspace{0.5cm} [Cr, Sr]\\
\indent\hspace{0.5cm}[(dowh$\cdot$((while E C)$\cdot$Cr)), ((A$\cdot$B)$\cdot$Sr)]\hspace{0.3cm} $\Rightarrow$ \hspace{0.5cm} [cons$^{*}$ C (while E C) Cr, Sr]\\
\indent\hspace{0.5cm}[nil, St]\hspace{3.9cm} $\Rightarrow$ \hspace{0.5cm} [nil, St]\\

\section{Constant in the Expression}\label{appendix:const of e}
First of all, we define a function that would construct some value in the format of \md{E} based on the data of \md{D}:
\begin{code}
dtoE : {n : ℕ} → D → E n
dtoE dnil = nil
dtoE (d₁ ∙ d₂) = cons (dtoE d₁) (dtoE d₂)
\end{code}
Then, there are sixteen constant in \md{E} that are used to indicate special meaning in the universal \WHILE program.
\begin{code}[fontsize=\footnotesize]
quoteE : {n : ℕ} → E n
quoteE = dtoE dquote

varE : {n : ℕ} → E n
varE = dtoE dvar

valueE : {n : ℕ} → (f : Fin n) → E n
valueE f = dtoE (dftod f)

hdE : {n : ℕ} → E n
hdE = dtoE dhd

dohdE : {n : ℕ} → E n
dohdE = dtoE dohd

tlE : {n : ℕ} → E n
tlE = dtoE dtl

dotlE : {n : ℕ} → E n
dotlE = dtoE dotl

consE : {n : ℕ} → E n
consE = dtoE dcons

doconsE : {n : ℕ} → E n
doconsE = dtoE docons

=?E : {n : ℕ} → E n
=?E = dtoE d=?

do=?E : {n : ℕ} → E n
do=?E = dtoE do=?

→→E : {n : ℕ} → E n
→→E = dtoE d→→

:=E : {n : ℕ} → E n
:=E = dtoE d:=

doasgnE : {n : ℕ} → E n
doasgnE = dtoE doasgn

whileE : {n : ℕ} → E n
whileE = dtoE dwhile

dowhE : {n : ℕ} → E n
dowhE = dtoE dowh
\end{code}

