\chapter{Definition of \WHILE Language}
The definition of \WHILE language used in the project are following the definition in the paper \wit{Computability and complexity: from a programming perspective} by \wit{Jones, Neil D} in 1997\cite{jones_computability_1997}.
\section{Syntax of Expression}\label{appendix:syntax of expression}
Expressions \hspace{0.3cm}$\ni$\hspace{0.3cm} E, F\hspace{0.1cm} ::= X \hspace{0.98cm}(for X $\in$ Vars)\\
\indent\hspace{4.2cm}$|$ $d$\hspace{1cm} (for atom $d$, one atom \wit{nil} defined in \Agda)\\
\indent\hspace{4.2cm}$|$ \textbf{cons} E F\\
\indent\hspace{4.2cm}$|$ \textbf{hd} E\\
\indent\hspace{4.2cm}$|$ \textbf{tl} E\\
\indent\hspace{4.2cm}$|$ \textbf{=?} E F\\
\section{Syntax of Command}\label{appendix:syntax of command}
Commands \hspace{0.35cm}$\ni$\hspace{0.35cm} C, D\hspace{0.1cm} ::= X := E\\
\indent\hspace{4.2cm}$|$ C \textbf{;} D\\
\indent\hspace{4.2cm}$|$ \textbf{while} E \textbf{do} C\\
\section{Semantics of Expression}\label{appendix:semantics of expression}
The definition of evaluation function $\upvarepsilon$ is: for $e \in \mathds{E}$ and a given \wit{store} of program $\mathds{P}$, $\sigma \in Store^{\mathds{P}}$, $\upvarepsilon \llbracket e \rrbracket \sigma = d \in \mathds{D}$.\\
\indent\hspace{3cm}$\varepsilon \rrbracket$X$\rrbracket\sigma$ \hspace{1.5cm} = \hspace{0.3cm} $\sigma($X$)$\\
\indent\hspace{3cm}$\varepsilon \llbracket$d$\rrbracket\sigma$ \hspace{1.58cm} = \hspace{0.3cm} d\\
\indent\hspace{3cm}$\varepsilon \llbracket$cons E F$\rrbracket\sigma$ \hspace{0.3cm} = \hspace{0.3cm} $\varepsilon \llbracket$E$\rrbracket\sigma \centerdot \varepsilon \llbracket$F$\rrbracket\sigma$ \\
\indent\hspace{3cm}$\varepsilon \llbracket$hd E$\rrbracket\sigma$ \hspace{1cm} = \hspace{0.3cm} $\begin{cases} e & \mbox{if } \varepsilon \llbracket E \rrbracket\sigma = (e,f)\\ \mbox{nil} & \mbox{otherwise}\end{cases}$\\
\indent\hspace{3cm}$\varepsilon \llbracket$tl E$\rrbracket\sigma$ \hspace{1.2cm} = \hspace{0.3cm} $\begin{cases} f & \mbox{if } \varepsilon \llbracket E \rrbracket\sigma = (e,f)\\ \mbox{nil} & \mbox{otherwise}\end{cases}$\\
\indent\hspace{3cm}$\varepsilon \llbracket$=? E F$\rrbracket\sigma$ \hspace{0.6cm} = \hspace{0.3cm} $\begin{cases} \mbox{true} & \mbox{if } \varepsilon \llbracket E\rrbracket\sigma = \varepsilon \llbracket$F$\rrbracket\sigma\\ \mbox{false} & \mbox{otherwise}\end{cases}$
\section{Semantics of Command}\label{appendix:semantics of command}
The definition of the execution relationship is: for \wit{c} $\in \mathds{C}$, $c \vdash \sigma \rightarrow \sigma' \subseteq \mathds{C} \times Store^{\mathds{P}} \times Store^{\mathds{P}}$ where $\sigma'$ is the new \wit{environment} updated by the execution of command \wit{c}.\\
X\textbf{:=}E $\vdash\sigma \rightarrow \sigma[$X $\mapsto$ d$]$ \hspace{1.15cm}if \hspace{0.2cm} $\varepsilon \llbracket E \rrbracket \sigma = $ d\\
C\textbf{;}D $\vdash\sigma \rightarrow \sigma''$ \hspace{2.8cm}if \hspace{0.2cm} C $\vdash\sigma \rightarrow \sigma'$ and D $\vdash\sigma' \rightarrow \sigma''$\\
\textbf{while} E \textbf{do} C $\vdash\sigma \rightarrow \sigma''$ \hspace{0.95cm}if \hspace{0.2cm} C $\varepsilon \llbracket E \rrbracket \sigma \neq $ nil, C $\vdash\sigma \rightarrow \sigma'$,\\ \indent\hspace{8.3cm}\textbf{while} E \textbf{do} C $\vdash\sigma' \rightarrow \sigma''$\\
\textbf{while} E \textbf{do} C $\vdash\sigma \rightarrow \sigma$ \hspace{1.15cm}if \hspace{0.2cm} C $\varepsilon \llbracket E \rrbracket\sigma = $ nil\\

\section{Example \WHILE program}\label{appendix:while example}
Here we give an example code on \WHILE language and use the defined part to simulate the program. 
\subsection{Example \WHILE program in \WHILE}
The \WHILE program \textbf{concat} which could concatenate two list into one define as below:
\begin{code}
read X; (* X is (d.e) *)
  A := hd X; (* A is d *)
  Y := tl X; (* Y is e *)
  B := nil; (* B becomes d reversed *)
  while A do
    B := cons (hd A) B;
    A := tl A;
  while B do
    Y := cons (hd B) Y;
    B := tl B;
write Y
\end{code}
\subsection{Example \WHILE program in \Agda}
Here we construct the same program using the definition we defined in \textbf{Agda}, it should be the following format:
\begin{code}[fontsize=\footnotesize]
append : P 4
append = prog zero
             ((suc (suc zero) := hd (var zero)) 
	     →→
             (suc zero := tl (var zero)) 
	     →→
             (suc (suc (suc zero)) := nil) 
	     →→
             (while
               (var (suc (suc zero)))
               ((suc (suc (suc zero)) := 
			cons (hd (var (suc (suc zero)))) 
			     (var (suc (suc (suc zero))))) 
	       →→
               ((suc (suc zero)) := tl (var (suc (suc zero)))))) 
	       →→
             (while
               (var (suc (suc (suc zero))))
               ((suc zero := cons (hd (var (suc (suc (suc zero))))) 
				  (var (suc zero))) 
	     →→
               (suc (suc (suc zero)) := tl (var (suc (suc (suc zero))))))))
             (suc zero)
\end{code}
\subsection{Execution of the example \WHILE program in \Agda}
To run the program, we define three list (in format of \md{D}) in which \textbf{list1} and \textbf{list2} are the two input lists and \textbf{list3} is the result:
\begin{code}
list1 : D
list1 = ltod (1 :: 2 :: 3 :: [])

list2 : D
list2 = ltod (4 :: 5 :: 6 :: [])

list3 : D
list3 = ltod (1 :: 2 :: 3 :: 4 :: 5 :: 6 :: [])
\end{code}
Now we could execute the \textbf{WHILE} program using our definitions of syntax and semantics:
\begin{code}[fontsize=\footnotesize]
runAppend : ExecP append (list1 ∙ list2) list3
runAppend = terminate zero (suc zero)
                 {env = list1 ∙ list2 ∷ list3 ∷ dnil ∷ dnil ∷ []}
            (seq {env₁ = list1 ∙ list2 ∷ dnil ∷ dnil ∷ dnil ∷ []}
             assign
            (seq {env₁ = list1 ∙ list2 ∷ dnil ∷ list1 ∷ dnil ∷ []}
             assign
            (seq {env₁ = list1 ∙ list2 ∷ list2 ∷ list1 ∷ dnil ∷ []}
             assign
            (seq {env₁ = list1 ∙ list2 ∷ list2 ∷ list1 ∷ dnil ∷ []}
                 {env₂ = result}
                 {env₃ =  list1 ∙ list2 ∷ list3 ∷ dnil ∷ dnil ∷ []}
            (whilet {env₁ = list1 ∙ list2 ∷ list2 ∷ list1 ∷ dnil ∷ []}
                    {env₂ = list1 ∙ list2 ∷ list2 ∷
                            dsnd list1 ∷ dfst list1 ∙ dnil ∷ []}
                    {env₃ = result}
             tt 
            (seq assign assign)
            (whilet {env₁ = list1 ∙ list2 ∷ list2 ∷
                            dsnd list1 ∷ dfst list1 ∙ dnil ∷ []}
                    {env₂ = list1 ∙ list2 ∷ list2 ∷ dsnd (dsnd list1) ∷
                            dfst (dsnd list1) ∙ (dfst list1 ∙ dnil) ∷ []}
                    {env₃ = result}
             tt
            (seq assign assign)
            (whilet {env₁ = list1 ∙ list2 ∷ list2 ∷ dsnd (dsnd list1) ∷
                            dfst (dsnd list1) ∙ (dfst list1 ∙ dnil) ∷ []}
                    {env₂ = result}
                    {env₃ = result}
             tt
            (seq assign assign)
            (whilef tt))))
            (whilet {env₁ = result}
                    {env₂ = list1 ∙ list2 ∷ const 3 ∙ list2 ∷
                            dnil ∷ ltod (2 ∷ 1 ∷ []) ∷ []}
                    {env₃ = list1 ∙ list2 ∷ list3 ∷ dnil ∷ dnil ∷ []}
             tt
            (seq assign assign)
            (whilet {env₁ = list1 ∙ list2 ∷ const 3 ∙ list2 ∷
                            dnil ∷ ltod (2 ∷ 1 ∷ []) ∷ []}
                    {env₂ = list1 ∙ list2 ∷ const 2 ∙ (const 3 ∙ list2) ∷
                            dnil ∷ ltod (1 ∷ []) ∷ []}
                    {env₃ = list1 ∙ list2 ∷ list3 ∷ dnil ∷ dnil ∷ []}
             tt
            (seq assign assign)
            (whilet {env₁ = list1 ∙ list2 ∷ const 2 ∙ (const 3 ∙ list2) ∷
                            dnil ∷ ltod (1 ∷ []) ∷ []}
                    {env₂ = list1 ∙ list2 ∷ list3 ∷ dnil ∷ dnil ∷ []}
                    {env₃ = list1 ∙ list2 ∷ list3 ∷ dnil ∷ dnil ∷ []}
             tt
            (seq assign assign)
            (whilef tt))))))))
            where
               result : Vec D 4
               result = list1 ∙ list2 ∷ list2 ∷
                        dnil ∷ ltod (3 ∷ 2 ∷ 1 ∷ []) ∷ []
\end{code}